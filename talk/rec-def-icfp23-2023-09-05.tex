\documentclass[xcolor=dvipsnames,aspectratio=169]{beamer}
%\documentclass{article}
%\usepackage{beamerarticle}

% % With greetings from Guy Steele
% \makeatletter
%     \setlength\beamer@paperwidth{14.00cm}%
%     \setlength\beamer@paperheight{9.10cm}%
%     \geometry{papersize={\beamer@paperwidth,\beamer@paperheight}}
% \makeatother

\usetheme{metropolis}
\usepackage{substrate-background}

\setbeamercolor{normal text}{%
  fg=black,
}

\usepackage[T1]{fontenc}
%\usepackage[english]{babel}
\usepackage[utf8]{inputenc}
\usepackage{tikz}
%\usetikzlibrary{shapes,arrows}
\usetikzlibrary{positioning,calc,decorations,decorations.pathmorphing,decorations.markings,shapes.geometric,matrix,arrows.meta}
\usetikzlibrary{overlay-beamer-styles}
\usetikzlibrary{patterns}
\usepackage{hyperref}
\usepackage{mathtools}
\usepackage{amssymb}
\usepackage{listings}
\usepackage{booktabs}
\usepackage{txfonts}
\usepackage[normalem]{ulem}
\renewcommand{\ULthickness}{0.8pt} % was 0.4pt
\usepackage{array}
\usepackage[scaled=0.81]{beramono}
\usepackage{relsize}

\usepackage{multirow}
\usepackage{bigstrut}

% http://tex.stackexchange.com/a/85578/15107
\tikzset{onslide/.code args={<#1>#2}{%
  \only<#1>{\pgfkeysalso{#2}} % \pgfkeysalso doesn't change the path
}}
\makeatletter
\newenvironment<>{btHighlight}[1][]
{\begin{onlyenv}#2\begingroup\tikzset{bt@Highlight@par/.style={#1}}\begin{lrbox}{\@tempboxa}}
{\end{lrbox}\bt@HL@box[bt@Highlight@par]{\@tempboxa}\endgroup\end{onlyenv}}

\newcommand<>\btHL[1][]{%
  \only#2{\begin{btHighlight}[#1]\bgroup\aftergroup\bt@HL@endenv}%
}
\def\bt@HL@endenv{%
  \end{btHighlight}%   
  \egroup
}
\newcommand{\bt@HL@box}[2][]{%
  \tikz[#1]{%
    \pgfpathrectangle{\pgfpoint{1pt}{0pt}}{\pgfpoint{\wd #2}{\ht #2}}%
    \pgfusepath{use as bounding box}%
    \node[anchor=base west,
    %fill=orange!30,
    outer sep=0pt,inner xsep=1pt, inner ysep=0pt, rounded corners=3pt, minimum height=\ht\strutbox+1pt,
    #1]{\raisebox{1pt}{\strut}\strut\usebox{#2}};
  }%
}
\makeatother



\lstdefinestyle{Haskell}{language=Haskell
        ,columns=flexible
	,basewidth={.365em}
	,keepspaces=True
        ,texcl=true
%        ,escapechar=!
        ,basicstyle=\ttfamily
        ,stringstyle=\small\ttfamily
	,keywordstyle=\color{OliveGreen}\bfseries
        ,showstringspaces=false
	,commentstyle=\sffamily
         ,literate=
%                   {->}{$\,\to\,$}2
%                   {<-}{$\,\leftarrow\,$}2
%                   {=>}{$\,\Rightarrow\,$}2
%                   {→}{$\,\to\,$}2
%                  {\\}{\textlambda}1
%                  {>>}{{>>}\hspace{-1pt}}2
%                  {+}{{$+$}}1
                  {[]}{[\,]\ }1
%                  {--}{{---\ }}1  would be nice, but prevents detection of comments
                  {++}{{$+\!\!+$}}1
%                 {\ .\ }{{$\,\circ\,$}}2
                  {STRUT}{{\strut}}1 % hack
                  {DOTS}{{\ldots}}1 % hack
                  {~}{{$\sim$}}1 % hack
                  {|>}{{$\triangleright$}}1 % hack
                  {ghci>}{{\bfseries\color{OliveGreen}$\lambda$>}}2
		  {PRAGMA}{\{-\#}3
		  {ENDPRAGMA}{\#-\}}3
	% Nicer comment separator
	% ,deletecomment=[l]--%
	% for RULEs
        ,deletecomment=[n]{\{-}{-\}}%
	,keywords={%
	    case,class,data,default,deriving,do,else,%
	    foreign,if,import,as,in,infix,infixl,infixr,instance,%
	    let,forall,letrec,module,newtype,role,of,then,type,where,\_,%
            nominal,representational,phantom,RULES,%
	}%
	,otherkeywords={%
	    {=>},{::},{->},{<-}%
            %,{=}%
        }%
        }
\lstdefinestyle{Rules}{
        ,style=Haskell
        ,deletecomment=[l]--%
        ,morekeywords={RULES}%
        }
\lstdefinestyle{gen}{
	frame=lr,framerule=2pt, rulecolor=\color{orange}
	}
\lstset{style=Haskell}
\newcommand{\li}{\lstinline[breaklines=true, breakatwhitespace=true]}

% \pause mit verstecken
\newcommand{\hide}{\onslide+<+(1)->}



%% From https://tex.stackexchange.com/a/50251/15107
\makeatletter
\newcommand*\Alt{\alt{\Alt@branch0}{\Alt@branch1}}

\newcommand\Alt@branch[3]{%
  \begingroup
  \ifbool{mmode}{%
    \mathchoice{%
      \Alt@math#1{\displaystyle}{#2}{#3}%
    }{%
      \Alt@math#1{\textstyle}{#2}{#3}%
    }{%
      \Alt@math#1{\scriptstyle}{#2}{#3}%
    }{%
      \Alt@math#1{\scriptscriptstyle}{#2}{#3}%
    }%
  }{%
    \sbox0{#2}%
    \sbox1{#3}%
    \Alt@typeset#1%
  }%
  \endgroup
}

\newcommand\Alt@math[4]{%
  \sbox0{$#2{#3}\m@th$}%
  \sbox1{$#2{#4}\m@th$}%
  \Alt@typeset#1%
}

\newcommand\Alt@typeset[1]{%
  \ifnumcomp{\wd0}{>}{\wd1}{%
    \def\setwider   ##1##2{##2##1##2 0}%
    \def\setnarrower##1##2{##2##1##2 1}%
  }{%
    \def\setwider   ##1##2{##2##1##2 1}%
    \def\setnarrower##1##2{##2##1##2 0}%
  }%
  \sbox2{}%
  \sbox3{}%
  \setwider2{\wd}%
  \setwider2{\ht}%
  \setwider2{\dp}%
  \setnarrower3{\ht}%
  \setnarrower3{\dp}%
  \leavevmode
  \rlap{\usebox#1}%
  \usebox2%
  \usebox3%
}
\makeatother
%% END https://tex.stackexchange.com/a/50251/15107


\usepackage{textgreek} % \textalpha etc.

%\DeclareTextSymbol\textlambda{T3}{171}           % Lambda
%\DeclareTextSymbolDefault\textlambda{T3}
\DeclareUnicodeCharacter{03BB}{\textlambda}

\usepackage{pifont}
\newcommand{\cmark}{\textcolor{OliveGreen}{\ding{51}}}%
\newcommand{\xmark}{\textcolor{RedOrange}{\ding{55}}}%

\usepackage{fancybox}
\makeatletter
\newenvironment{CenteredBox}{% 
\begin{Sbox}}{% Save the content in a box
\end{Sbox}\centerline{\parbox{\wd\@Sbox}{\TheSbox}}}% And output it centered
\makeatother

\makeatletter
\newcommand\HUGE{\@setfontsize\Huge{45}{50}}
\makeatother


\newcommand{\shout}[1]{
\begin{frame}
\begin{center}
\HUGE #1
\end{center}
\end{frame}
}

\newcommand{\heart}{\textcolor{red}{\scalebox{1.6}{\ensuremath\varheartsuit}}}

\title{More Fixpoints! (Functional Pearl)}
\subtitle{Joachim Breitner}
\institute{}
\date{September 5, 2023\\ICFP, Seattle}
%\logo{\includegraphics{Telekom-Stiftung}}
%\titlegraphic{\vspace{6cm}\hfill\includegraphics[height=1.2cm]{logo-dark-text.pdf}}

\begin{document}

\frame[plain]{\titlepage}

% https://replicate.com/cjwbw/stable-diffusion-high-resolution
% seed 2690042245
% A lazy sloth hanging relaxed from a branch munging happily, Oil paining by Rembrandt
%\frame[plain]{\includegraphics[width=\linewidth]{sloth}}

%\section{Preaching to the choir}

\shout{I {\heart} Haskell}

\shout{I {\heart} Purity}

\shout{$\texttt{=} \mathrel{\textrm{=}} \mathrm{=}$}

\shout{I {\heart} Laziness}

\begin{frame}[fragile]{Laziness allows recursive definitions}
\strut
{\pause}

\lstset{language=C}
\lstset{moredelim=**[is][{\btHL<3>[name=show1, fill=orange,remember picture]}]{@1}{@}}
\lstset{moredelim=**[is][{\btHL<3>[name=show2, fill=orange,remember picture]}]{@2}{@}}
\lstset{moredelim=**[is][{\btHL<3>[name=show3, fill=orange,remember picture]}]{@3}{@}}
\lstset{moredelim=**[is][{\btHL<3>[name=show4, fill=orange,remember picture]}]{@4}{@}}
\begin{CenteredBox}
\begin{lstlisting}
bool @1is_even@(unsigned int n) {
    if (n == 0) return true;
    else        return @2is_odd@(n - 1);
}

bool @3is_odd@(unsigned int n) {
    if (n == 0) return false;
    else        return @4is_even@(n - 1);
}
\end{lstlisting}
\end{CenteredBox}
\only<3>{%
\tikz[overlay,remember picture]{\draw[->,bend right, color=orange] (show2) to (show3)}%
\tikz[overlay,remember picture]{\draw[->,bend left, color=orange] (show4) to (show1)}%
}%
~
\end{frame}

\begin{frame}[fragile]{C is lazy, but Haskell is lazier}
\begin{center}
\begin{tabular}{c|c|c}
C & Haskell & \phantom{Haskell}\\
 & & \phantom{+ rec-def}\\
\hline
\bigstrut
functions & functions        & \phantom{functions} \\
          & \lstinline-data- & \phantom{\lstinline-data-} \\
          &                  & \phantom{Sets, Booleans} \\
\end{tabular}
\end{center}
\end{frame}

\begin{frame}[fragile]{A famous tied knot}
\lstset{moredelim=**[is][{\btHL<2>[name=show1, fill=orange,remember picture]}]{@1}{@}}
\lstset{moredelim=**[is][{\btHL<2>[name=show2, fill=orange,remember picture]}]{@2}{@}}
\lstset{moredelim=**[is][{\btHL<2>[name=show3, fill=orange,remember picture]}]{@3}{@}}
\begin{CenteredBox}
\begin{lstlisting}
fibs :: [Integer]
@1fibs@ = 0 : 1 : zipWith (+) @2fibs@ (tail @3fibs@)
\end{lstlisting}
\end{CenteredBox}
\only<2>{%
\tikz[overlay,remember picture]{\draw[->,bend left, color=orange] (show2.south west) to (show1.south east)}%
\tikz[overlay,remember picture]{\draw[->,bend left, color=orange] (show3.south west) to (show1.south east)}%
}%
~
\end{frame}

\begin{frame}[fragile,t]{The reflexive transitive closure}
\begin{lstlisting}
type Graph = M.Map Int [Int]

rtrans :: Graph -> Graph
rtrans g = M.map S.toList reaches
  where
    reaches :: M.Map Int (S.Set Int)
    reaches = M.mapWithKey f g
    f v vs = S.insert v (S.unions [ reaches M.! v' | v' <- vs ]))
\end{lstlisting}
\pause
\begin{center}
\lstinline-rtrans(-
\begin{tikzpicture}[anchor=base,baseline]
\node[draw,circle, inner sep=1pt] (1) at (0,0) {1};
\node[draw,circle, inner sep=1pt] (2) at (1,0) {2};
\node[draw,circle, inner sep=1pt] (3) at (2,0) {3};
\path[draw,->] (1) to [bend right] (3);
\path[draw,->] (2) to [bend right] (1);
\path<3->[thick,red,draw,->] (1) to [bend right] (2);
\end{tikzpicture}
\lstinline-) = -
\Alt<3->{\visible<4->{$\bot$}}{%
\begin{tikzpicture}[anchor=base,baseline]
\node[draw,circle, inner sep=1pt] (1) at (0,0) {1};
\node[draw,circle, inner sep=1pt] (2) at (1,0) {2};
\node[draw,circle, inner sep=1pt] (3) at (2,0) {3};
\path[draw,->] (1) to [bend right] (3);
\path[draw,->] (1) to [loop above] (1);
\path[draw,->] (2) to [bend right] (1);
\path[draw,->] (2) to [loop above] (2);
\path[draw,->] (2) to [bend left] (3);
\path[draw,->] (3) to [loop above] (3);
\end{tikzpicture}
}
\end{center}
\end{frame}

\begin{frame}[fragile,t]{What happened here?}
\begin{lstlisting}
rtrans $ M.fromList [(1,[2,3]),(2,[1]),(3,[])]


rtrans g = M.map S.toList reaches
  where
    reaches :: M.Map Int (S.Set Int)
    reaches = M.mapWithKey f g
    f v vs = S.insert v (S.unions [ reaches M.! v' | v' <- vs ])
\end{lstlisting}
\end{frame}

\begin{frame}[fragile,t]{What happened here?}
\begin{lstlisting}



M.map S.toList reaches
  where
    reaches :: M.Map Int (S.Set Int)
    reaches = M.mapWithKey f g
    f v vs = S.insert v (S.unions [ reaches M.! v' | v' <- vs ])
    g = M.fromList [(1,[2,3]),(2,[1]),(3,[])]
\end{lstlisting}
\end{frame}

\begin{frame}[fragile,t]{What happened here?}
\begin{lstlisting}



M.map S.toList reaches
  where
    reaches :: M.Map Int (S.Set Int)
    reaches = M.mapWithKey f (M.fromList [(1,[2,3]),(2,[1]),(3,[])])
    f v vs = S.insert v (S.unions [ reaches M.! v' | v' <- vs ])
\end{lstlisting}
\end{frame}

\begin{frame}[fragile,t]{What happened here?}
\begin{lstlisting}



M.map S.toList reaches
  where
    reaches :: M.Map Int (S.Set Int)
    reaches = M.fromList [(1, f 1 [2,3]),(2,f 2 [1]),(3, f 3 [])]
    f v vs = S.insert v (S.unions [ reaches M.! v' | v' <- vs ])
\end{lstlisting}
\end{frame}

\begin{frame}[fragile,t]{What happened here?}
\begin{lstlisting}



M.map S.toList reaches
  where
    reaches :: M.Map Int (S.Set Int)
    reaches = M.fromList [(1, s1),(2, s2),(3, s3)]
    f v vs = S.insert v (S.unions [ reaches M.! v' | v' <- vs ])
    s1 = f 1 [2,3]
    s2 = f 2 [1]
    s3 = f 3 []
\end{lstlisting}
\end{frame}

\begin{frame}[fragile,t]{What happened here?}
\begin{lstlisting}



M.map S.toList reaches
  where
    reaches :: M.Map Int (S.Set Int)
    reaches = M.fromList [(1, s1),(2, s2),(3, s3)]

    s1 = S.insert 1 (S.unions [ reaches M.! v' | v' <- [2,3] ]))
    s2 = S.insert 2 (S.unions [ reaches M.! v' | v' <- [1] ]))
    s3 = S.insert 3 (S.unions [ reaches M.! v' | v' <- [] ]))
\end{lstlisting}
\end{frame}

\begin{frame}[fragile,t]{What happened here?}
\begin{lstlisting}



M.map S.toList reaches
  where
    reaches :: M.Map Int (S.Set Int)
    reaches = M.fromList [(1, s1),(2, s2),(3, s3)]

    s1 = S.insert 1 (S.unions [ reaches M.! 2, reaches M.! 3])
    s2 = S.insert 2 (S.unions [ reaches M.! 1 ])
    s3 = S.insert 3 (S.unions [])
\end{lstlisting}
\end{frame}

\begin{frame}[fragile,t]{What happened here?}
\begin{lstlisting}



M.map S.toList reaches
  where
    reaches :: M.Map Int (S.Set Int)
    reaches = M.fromList [(1, s1),(2, s2),(3, s3)]

    s1 = S.insert 1 (S.unions [ s2, s3 ])
    s2 = S.insert 2 (S.unions [ s1 ])
    s3 = S.insert 3 (S.unions [ ])
\end{lstlisting}
\end{frame}

\begin{frame}[fragile,t]{What happened here?}
\lstset{moredelim=**[is][{\btHL<2>[name=show1, fill=orange,remember picture]}]{@1}{@}} \lstset{moredelim=**[is][{\btHL<2>[name=show2, fill=orange,remember picture]}]{@2}{@}}
\lstset{moredelim=**[is][{\btHL<2>[name=show3, fill=orange,remember picture]}]{@3}{@}}
\lstset{moredelim=**[is][{\btHL<2>[name=show4, fill=orange,remember picture]}]{@4}{@}}
\begin{lstlisting}



M.map S.toList (M.fromList [(1,s1),(2,s2),(3,s3)])
  where



    @1s1@ = S.insert 1 (S.unions [ @2s2@, s3 ])
    @3s2@ = S.insert 2 (S.unions [ @4s1@ ])
    s3 = S.insert 3 (S.unions [ ]))
\end{lstlisting}%
\only<2>{%
\tikz[overlay,remember picture]{\draw[->,bend right, color=orange] (show2) to (show3)}%
\tikz[overlay,remember picture]{\draw[->,bend right, color=orange] (show4) to (show1)}%
}%
\end{frame}

\begin{frame}[fragile]{The set API}
\begin{lstlisting}
import Data.Set as S
data Set a
S.insert  :: Ord a => a -> Set a -> Set a    -- too strict
S.unions  :: Ord a => [Set a] -> Set a       -- too strict
\end{lstlisting}
\pause
\begin{lstlisting}
import Data.Recursive.Set as RS              -- from rec-def
data RSet a
RS.insert :: Ord a => a -> RSet a -> RSet a  -- lazy enough
RS.unions :: Ord a => [RSet a] -> RSet a     -- lazy enough
RS.get    :: RSet a -> Set a
\end{lstlisting}
\end{frame}

\begin{frame}[fragile,t]{The reflexive transitive closure, using \phantom{R}Set}
\begin{lstlisting}


rtrans :: Graph -> Graph
rtrans g = M.map (S.toList         ) reaches
  where
    reaches :: M.Map Int (  S.Set Int)
    reaches = M.mapWithKey f g
    f v vs =  S.insert v ( S.unions [ reaches M.! v' | v' <- vs ]))
\end{lstlisting}
\end{frame}

\begin{frame}[fragile,t]{The reflexive transitive closure, using RSet}
\lstset{moredelim=**[is][{\btHL<1>[name=show4, opacity=.8, text opacity=1, fill=yellow,remember picture]}]{@}{@}}
\begin{lstlisting}
import qualified RSet as RS

rtrans :: Graph -> Graph
rtrans g = M.map (S.toList@ . RS.get@) reaches
  where
    reaches :: M.Map Int (@RS.R@Set Int)
    reaches = M.mapWithKey f g
    f v vs = @RS@.insert v (@RS@.unions [ reaches M.! v' | v' <- vs ]))
\end{lstlisting}
\end{frame}

\begin{frame}[fragile]{Is this still kosher?}
\begin{itemize}
\item The library finds a solution to the equation. Given
\begin{lstlisting}
let s1 = RS.insert 1 (RS.unions [ s2, s3 ])
\end{lstlisting}
we get
\begin{lstlisting}
RS.get s1 == S.insert 1 (S.unions [ RS.get s2, RS.get s3 ])
\end{lstlisting}
Equational reasoning still works!
\item Sets (with monotonic functions) are nice:

\medskip

Unique least solution always exists
\end{itemize}
\end{frame}

\begin{frame}[fragile]{My main claim}
\begin{center}
\Large
When sets of equations have a unique solution,

\bigskip

they should work as recursive definitions.
\end{center}
\end{frame}


\begin{frame}[fragile]{C is lazy, but Haskell is lazier, could be even more lazy}
\begin{center}
\begin{tabular}{c|c|c}
C & Haskell & {Haskell}\\
 & & {(but more lazy)}\\
\hline
\bigstrut
functions & functions        & {functions} \\
          & \lstinline-data- & {\lstinline-data-} \\
          &                  & \lstinline-RSet- \\
	  &                  & \lstinline-RBool- \\
          &                  & $\mathbb{N}$ \\
          &                  & grammars \\
          &                  & \ldots \\
\end{tabular}
\end{center}
\end{frame}

\begin{frame}[fragile]{What's the gain?}
\begin{itemize}
\item More problems solved by pure, declarative code
\item Separates declaration and solving
\item Avoids fixed-point combinators or monads
\item Particularly tricky with combinators:
\begin{itemize}
\item Mutual recursion
\item Mutual recursion at different types
\item Many variables declared afar, but solved in one go
\end{itemize}
\end{itemize}
\pause
\bigskip
\begin{center}
\lstinline-let- is best to describe recursive equations
\end{center}
\end{frame}

\begin{frame}[fragile]{What's the cost?}
\begin{itemize}
\item Just as mind-bending as tying-the-knot techniques for sharing
\item Only applicable in domains where unique solution exists and is computable
\item Lambda-lifting can affect termination:
\begin{lstlisting}
ghci> let s1    = RS.insert 42  s1     in RS.get  s1
fromList [42]
ghci> let s1 () = RS.insert 42 (s1 ()) in RS.get (s1 ())
fromList ^CInterrupted.
\end{lstlisting}
\end{itemize}
\end{frame}

\begin{frame}{What's in the paper?}
\begin{itemize}
\item Larger use-case: program analysis
\item Same type, different orders
\item Implementation insights:\\
\begin{itemize}
\item From imperative propagator libraries to pure API
\item \lstinline-unsafePerformIO-
\item Sharing as a source of uniqueness
\item Concurrency, reentrancy and space leak concerns
\item The \lstinline-let x = x- issue
\end{itemize}
\item Denotational semantics (WIP draft)
\item Try it at \url{https://hackage.haskell.org/package/rec-def}
\end{itemize}
\end{frame}

\begin{frame}{Thank you for your attention!}
\begin{center}
{\Huge
$\texttt{=} \mathrel{\textrm{=}} \mathrm{=}$

}
\vfill
\hrule
\vfill
\Large
When sets of equations have a unique solution,

\bigskip

they should work as recursive definitions.
\vfill
\end{center}
\end{frame}

\appendix
\section{Backup slides}

\begin{frame}[fragile]{API excerpt}
We have
\begin{lstlisting}
RS.mk           :: Set a -> RSet a
RS.insert       :: Ord a => a -> RSet a -> RSet a
RS.delete       :: Ord a => a -> RSet a -> RSet a
RS.union        :: Ord a => RSet a -> RSet a -> RSet a
RS.intersection :: Ord a => RSet a -> RSet a -> RSet a
RS.member       :: Ord a => a -> RSet a -> RBool
RB.&&           :: RBool -> RBool -> RBool
\end{lstlisting}
We do not have
\begin{lstlisting}
RS.map          :: Ord b => (a -> b) -> RSet a -> RSet b  -- would be okish
RS.difference   :: Ord a => RSet a -> RSet a -> RSet a    -- would be bad
RB.not          :: RBool -> RBool                         -- would be bad
\end{lstlisting}
\end{frame}



\section{So how does it work?}

\begin{frame}[fragile]{Breaking down the problem}
\begin{enumerate}
\item A monadic “propagator”\\
(declare cells, declare relationships, solves, read values)
\item The pure wrapping
\item Some issues we gloss over today
\end{enumerate}

\pause

Our (simplified) goal:

\begin{lstlisting}
data RSet a
insert :: a -> RSet a -> RSet a
get    :: RSet a -> Set a
\end{lstlisting}

\end{frame}

\begin{frame}[fragile]{The propagator -- the API}
\begin{lstlisting}[basicstyle=\ttfamily\relscale{0.95}]
data Cell a
newC    :: IO (Cell a)

insertC :: Ord a => Cell a -> a -> Cell a -> IO ()









getC    :: Cell a -> IO (Set a)
STRUT
\end{lstlisting}
\end{frame}

\begin{frame}[fragile]{The propagator -- a naive(!) implementation}
\begin{lstlisting}[basicstyle=\ttfamily\relscale{0.95}]
data Cell a = C (IORef (Set a)) (IORef [IO ()])
newC    :: IO (Cell a)
newC = C <$> newIORef S.empty <*> newIORef []
insertC :: Ord a => Cell a -> a -> Cell a -> IO ()
insertC (C s0 ws0) x (C s1 ws1) = do
  let update = do
        new <- S.insert x <$> readIORef s1
        old <- readIORef s0
        unless (old == new) $ do
          writeIORef s0 new
          readIORef ws0 >>= sequence_
  modifyIORef ws1 (update :)
  update
getC    :: Cell a -> IO (Set a)
getC (C s1 _) = readIORef s1
\end{lstlisting}
\end{frame}


\begin{frame}[fragile]{The pure wrapper -- the API}
\begin{lstlisting}
data RSet a

insert :: Ord a => a -> RSet a -> RSet a








get :: RSet a -> Set a
STRUT
STRUT
\end{lstlisting}
\end{frame}

\begin{frame}[fragile]{The pure wrapper -- let's get dirty}
\begin{center}
\lstinline!unsafePerformIO :: IO a -> a!
\end{center}
\end{frame}

\begin{frame}[fragile]{A thunking data structure}
\begin{lstlisting}
data DoOnce

later :: IO () -> IO DoOnce


doNow :: DoOnce -> IO ()
STRUT
STRUT
STRUT
STRUT
STRUT
\end{lstlisting}
\end{frame}

\begin{frame}[fragile]{A thunking data structure}
\begin{lstlisting}
data DoOnce = DoOnce (IO ()) (IORef Bool)

later :: IO () -> IO DoOnce
later act = DoOnce act <$> newIORef False

doNow :: DoOnce -> IO ()
doNow (DoOnce act done) = do
    is_done <- readIORef done
    unless is_done $ do
        writeIORef done True
        act
\end{lstlisting}
\end{frame}

\begin{frame}[fragile]{The pure wrapper -- a naive(!) implementation}
\begin{lstlisting}
data RSet a = RSet (Cell a) DoOnce

insert :: Ord a => a -> RSet a -> RSet a
insert x r2 = unsafePerformIO $ do
  c1 <- newC
  todo <- later $ do
      let (RSet c2 todo2) = r2
      insertC c1 x c2
      doNow todo2
  return (RSet c1 todo)

get :: RSet a -> Set a
get (RSet c todo) = unsafePerformIO $ do
  doNow todo >> getC c
\end{lstlisting}
\end{frame}



\begin{frame}[fragile]{Simplified for your viewing pleasure}
\begin{itemize}
\item Other data types\\
\lstinline-RBool- with \lstinline!(RB.&&)! etc.
\item Mixing different data types\\
\lstinline!RS.member :: Ord a => a -> RSet a -> RBool!
\item Concurrency and reentrancy issues (\lstinline-unsafePerformIO-!)
\item Space leaks (watchers!)
\end{itemize}
\end{frame}


\iffalse
\begin{frame}[fragile]{Not \emph{all} equations}
Ok:
\begin{lstlisting}
let b = b RB.&& b
let s = S.insert 42 s
let s = S.singleton 42 `RS.union` s
\end{lstlisting}

\pause

Not ok:
\begin{lstlisting}
let b = RB.not b
let s = RS.singleton 42 `RS.difference` s
\end{lstlisting}
\end{frame}
\fi

\section{Is this still Haskell?}

\begin{frame}{Queasy about unsafePerformIO?}
\begin{quote}
that is why the function
is unsafe.\\
However “unsafe” is not the same as “wrong”. It simply means that
the programmer, not the compiler, must undertake the proof obligation that the
program’s semantics is unaffected [\ldots]
\end{quote}
\footnotesize
\raggedleft
“Stretching the Storage Manager: Weak Pointers and Stable Names in Haskell”\\
Simon Peyton Jones, Simon Marlow, and Conal Elliott
\end{frame}

\begin{frame}{Is this still Haskell?}
\begin{itemize}
\item Type safety \cmark
\item Independence of evaluation order \cmark\\
{\footnotesize (At least if the \emph{ascending chain conditions} holds, else unclear.)}
\item Equational reasoning \cmark\\
\lstinline-let x = E1[x] in E2[x]- $\quad \equiv\quad $ \lstinline-let x = E1[x] in E2[E1[x]]-
\lstinline-let x = E1[x] in E2[x]- $\quad \equiv\quad $ \lstinline-let x = E1[y]; y = E1[x] in E2[x]-
\pause
\item Lambda lifting \xmark \\
\lstinline-let x = E1[x,e] in E2[x]- $\quad \not\equiv\quad $ \lstinline-let x y = E1[x y,y] in E2[x e]-
Transformations that break \emph{sharing} can prevent termination!\\
{\footnotesize (So far: can only increase costs, but otherwise unobservable)}
\end{itemize}
\pause
\begin{center}
\ldots and how to prove it?
\end{center}
\end{frame}

\section{Theory}
\begin{frame}[fragile]{All involved functions must be \emph{monotone}}
For sets:
\begin{center}
If $s_1 \subseteq s_2$ then $f(s_1) \subseteq f(s_2)$.
\end{center}

For \lstinline-Bool-:
\begin{center}
If $b_1 \le b_2$ then $f(b_1) \le f(b_2)$.
\end{center}
where \lstinline-False- $\le$ \lstinline-True-.

\end{frame}

\begin{frame}{Finding the least fixed-point}
Let $X$ be partially ordered by $\sqsubseteq$, $\bot\in X$ be its least element, and $f \colon X \to X$ be a continuous function (i.e. $x \sqsubseteq y \implies f(x) \sqsubseteq f(y)$).

\pause

Then the sequence
\[
\bot \sqsubseteq f(\bot) \sqsubseteq f(f(\bot)) \sqsubseteq \ldots
\]
either diverges (all elements are different), or eventually finds a least fixed-point $x\in X$ of $f$, where
\[
x = f(x).
\]
\pause
If $X$ has the \emph{Ascending Chain Condition} (i.e. no infinite chain $x_0 \sqsubset x_1 \sqsubset \ldots$ exists), then the fixed-point will always be found.
\end{frame}

\section{A program analysis example}

\begin{frame}[fragile]{A small programming language}
\begin{lstlisting}
type Var = String

data Exp
  = Var Var
  | Lam Var Exp
  | App Exp Exp
  | Throw
  | Catch Exp
  | Let Var Exp Exp
\end{lstlisting}
\end{frame}

\begin{frame}[fragile]{A small analysis}
\begin{lstlisting}
canThrow1 :: Exp -> Bool
canThrow1 = go M.empty where
    go :: M.Map Var Bool -> Exp -> Bool
    go env (Var v)       = env M.! v
    go env Throw         = True
    go env (Catch e)     = False
    go env (Lam v e)     = go (M.insert v False env) e
    go env (App e1 e2)   = go env e1 || go env e2
    go env (Let v e1 e2) = go env' e2 where
        env_bind = M.fromList [ (v, go env e1) ]
        env' = M.union env_bind env
\end{lstlisting}
\end{frame}

\begin{frame}[fragile]{Let's add recursion}
\begin{lstlisting}
data Exp
  ...
  | LetRec [(Var, Exp)] Exp
\end{lstlisting}
\pause 
\lstset{moredelim=**[is][{\btHL<3>[name=show, fill=orange,remember picture]}]{@}{@}}
\begin{lstlisting}
canThrow1 :: Exp -> Bool
canThrow1 = go M.empty where
    go :: M.Map Var Bool -> Exp -> Bool
    ...
    go env (LetRec binds e) = go env' e where
        env_bind = M.fromList [ (v, go @env'@ e) | (v,e) <- binds ]
        env' = M.union env_bind env
\end{lstlisting}
\end{frame}

\begin{frame}[fragile,t]{Again, this fails with cyclic values}
\begin{lstlisting}
ghci> someVal = Lam "y" (Var "y")
ghci> prog = LetRec [("x", App (Var "x") someVal), ("y", Throw)] (Var "x")
ghci> canThrow1 prog
^CInterrupted.
\end{lstlisting}
\end{frame}

\begin{frame}[fragile,t]{Data.Recursive.Bool API}
\begin{lstlisting}
import Data.Recursive.Bool as RB

RB.true  :: RBool
RB.false :: RBool
RB.&&    :: RBool -> RBool -> RBool
RB.||    :: RBool -> RBool -> RBool
RB.and   :: [RBool] -> RBool
RB.or    :: [RBool] -> RBool
DOTS
RB.get   :: RBool -> Bool
\end{lstlisting}
\end{frame}

\newsavebox{\canThrowiiBox}
\begin{lrbox}{\canThrowiiBox}%
\begin{lstlisting}
canThrow2 :: Exp -> Bool
canThrow2 = RB.get . go M.empty where
    go :: M.Map Var RBool -> Exp -> RBool
    go env (Var v)       = env M.! v
    go env Throw         = RB.false
    go env (Catch e)     = RB.true
    go env (Lam v e)     = go (M.insert v RB.false env) e
    go env (App e1 e2)   = go env e1 RB.|| go env e2
    go env (Let v e1 e2) = go env' e2 where
        env_bind = M.singleton v (go env e1)
        env' = M.union env_bind env
    go env (LetRec binds e) = go env' e where
        env_bind = M.fromList [ (v, go env' e) | (v,e) <- binds ]
        env' = M.union env_bind env
\end{lstlisting}
\end{lrbox}

\begin{frame}[fragile,t]
\scalebox{1}{\usebox{\canThrowiiBox}}
\end{frame}

\begin{frame}[fragile,t]{Now it works!}
\begin{lstlisting}
ghci> someVal = Lam "y" (Var "y")
ghci> prog = LetRec [("x", App (Var "x") someVal), ("y", Throw)] (Var "x")
ghci> canThrow1 prog
^CInterrupted.
ghci> canThrow2 prog
False
\end{lstlisting}
\end{frame}


\end{document}

